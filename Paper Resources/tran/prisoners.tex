% LLNCS macro package for Springer Computer Science procedings;
% Version 2.20 of 2017/10/04
%\documentclass[runningheads]{llncs}
\documentclass[]{llncs}

%\usepackage{natbib} %chris for citep
%\usepackage{amsmath}  %chris for multiline stdp equation
%\usepackage{caption} %chris for table captions

%\pagestyle{plain} %undone remove these to get rid of page numbers
%\pagenumbering{arabic}

\begin{document}

\title{Prisoners}
%\titlerunning{Categorising with Spiking Nets}
\author{Chris Tcacik\inst{1}}
%\authorrunning{Huyck }
\institute{Middlesex University, London NW4 4BT UK\\
\email{c.huyck@mdx.ac.uk}\\
\url{https://cwa.mdx.ac.uk/chris/chrisroot.html}}

\maketitle              

%\today %put in to get draft date

\begin{abstract}
undone we need to write something


% doesn't latex at home
\keywords{The Tragic Valley: The Prisoners Dilemmas and Cooperation}
\end{abstract}

\section{Introduction}

The basic priosners dilemma is a well known phenomena.  \cite {Axelrod}
ran his well known experiments on the two person prisoner's dilemma... undone

The N-Person prisoners' dilemma involves $n$ people participating in the
dilemma.

The take home points are:
1. When prisoners only have one choice per turn, most algorithms descend into
the tragic valley.  When each prisoner has n-1 choices, it is relatively easy
to stay on top of the reciprocity hill.

2.  Contex is crucial in overcoming the tragic valley. History, value
(distance in lit).

3.  Adaptive MARL can rise out of the tragic valley.

4.  Optimistic tit for tat (and tit for tat) vs. standard reinforcement
learning descend into the valley.  

The human brain is the basis of intelligent behaviour, including
with biology (discussed in section \ref {secDiscussion}), but the overall
{secDiehl}.  The simulations use spiking neurons with dynamic
thresholds for some of the neurons.  A three population topology is

relatively easily using these mechanisms.  Biological plausibility
and future work are discussed in section \ref {secDiscussion}.

\section{Literature Review}
\label {secLitRev}

The work reported in this paper is the fourth in a series of papers
using biologically motivated simulated neurons and learning rules.
The first two papers in the series \cite {Huyck,Huyck-Samey} were
based on simulations that used a feed forward topology with input
neurons connected to category neurons.  The third was based on
competitive topology with three populations \cite
{HuyckErekpaine}. This topology and mechanism are derived from work by
Diehl and Cook \cite {Diehl}.  This is the basic topology, shown in
described below.


\section{Discussion}
\label {secDiscussion}

\section{Conclusion}

So, it is clear that these adaptive spiking neuron systems learning
with STDP can be used for categorisation.  This is not novel, but the
basics of this mechanism have been described above and it has been
extended to a novel digit recognition task. As a machine learning

neurobiology to see how it is done in brains and in petri dishes.  It
is quasi-neurobiologically plausible learning.


%\bibliographystyle{spbasic} % basic style, author-year citations
%\bibliographystyle{splncs04} %Bibliography names
\bibliographystyle{ieeetr}
\bibliography{prisoners}
\end{document}

